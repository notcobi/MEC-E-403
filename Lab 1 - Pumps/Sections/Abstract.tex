\begin{abstract}
    %$This will be a summary of the lab. It will include the purpose of the lab, the methods used, and the results obtained. This will be completed after the lab is finished.

Centrifugal pumps are used in many applications, including water supply, irrigation, and sewage systems. Understanding pump performance and key parameters is important for the design and operation of these systems.

This lab experiment was conducted to investigate the performance of a centrifugal pump. The pump was tested in single, parallel, and series configurations. The pump efficiency was also calculated.

Generally, the manufacturer's performance was higher than the experimental performance. This is illustrated in multiple plots, where the manufacturer specifications had higher head for a given flow rate. In addition, in the efficiency plot, the manufacturer's efficiency was higher than the experimental efficiency for all pump speeds. The history of the experimental pump was not known, and it is possible that the pump was not operating at its peak performance.

Shut off head was also analyzed. Two scenarios were considered: the rule of thumb shutoff head and the ideal shutoff head. The ideal case assumes that all kinetic energy is lost due to friction. The rule of thumb case assumes that half of the kinetic energy is lost due to friction. The ideal shutoff head was 48.3\% higher than the experimental shutoff head, while the rule of thumb shutoff head was within 3.3\% of the experimental shutoff head. This suggests good agreement between the experimental and rule of thumb shutoff head.

Uncertainties were calculated for all parameters. The largest source of error for all calculated parameters was the precision error from time. This was due to the reaction time of the stopwatch operator. To reduce this uncertainty, a calibrated digital flow meter should be used to measure the flow rate.
\end{abstract}