\begin{abstract}
\noindent Centrifugal pumps are used in many applications, including water supply, irrigation, and sewage systems. The general purpose of a pump is to increase the pressure of a fluid. Due to their wide application, the likelyhood of encountering a centrifugal pump in engineering design is significant. Understanding pump performance, limitations, and key parameters is important for the design and operation of these systems.

This lab experiment was conducted to investigate the performance of a centrifugal pump, comparing them to the manufacturer's specifications. The pump was tested in single, parallel, and series configurations at speeds of 1800, 2700, and 3600 RPM. The tests were performed by setting the pump in the desired configuration (eg. single, parallel, or series), and then measuring the time it took to fill 200 lb of water. The weight was measured with a scale, pressure measured with a transducer, torque measured with a dyno mass, and pump speed with a strobotach. 

Generally, the manufacturer's performance was higher than the experimental performance. This is illustrated in multiple plots, where the manufacturer specifications had higher head coefficient for a given flow coefficient, at most with an error of 39.7\% with respect to the manufacturer at the fully open 3600 RPM configuration. In addition, in the efficiency plot, the manufacturer's efficiency was higher than the experimental efficiency for all pump speeds. 

Parallel and series pump performance had poor agreement between the theoretical and experimental data. The theoretical head and flow had an error of 30.9\% and 33.7\% respectively for the parallel pump, and 23.1\% and 13.0\% respectively for the series pump. This suggests that the theoretical model does not accurately predict the parallel pump performance, and somewhat accurately predicts the series pump performance.

An analysis of the specification sheet, through plotting head coefficient - flow coefficient curves, suggested that the pump impeller width and height were not geometrically similar to the impeller diameter. For example, this suggests that the specification for a 96mm impeller diameter had the same width and height as a 102mm impeller diameter.

The investigation found that the performance of the pump was lower than the manufacturer's specifications. The history of the experimental pump was not known, and it is possible that the pump was not operating at its peak performance. It is critical to test the pump before designing a pump system, as the manufacturer's specifications may not be accurate. Further work in improving flow rate measurements through utilizing a calibrated digital flow meter should be considered to reduce the uncertainty in the results.
\end{abstract}