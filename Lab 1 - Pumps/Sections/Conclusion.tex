\section{Conclusion}
The focus of this lab was to investigate and understand the performance of centrifugal pumps. The pump performance was studied in single, parallel, and series configurations. The pump efficiency was also calculated. 

The head and flow coefficients were plotted for the experimental data. The head and flow coefficients for the single pump fell onto the same curve. The manufacturer's head and flow coefficients also fell onto a single, but different, curve. The manufacturer head coefficient was higher than the experimental head coefficient for the same flow coefficient. The ideal head and flow coefficients were linear and did not agree with the experimental and manufacturer data.

The rule of thumb shutoff head was within 3.3\% of the experimental shutoff head. This suggests good agreement between the experimental and rule of thumb shutoff head. The ideal shutoff head was 48.3\% higher than the experimental shutoff head. This suggests poor agreement between the experimental and ideal shutoff head.

The parallel pump performance had poor agreement between the theoretical and experimental data. The theoretical head was higher than the experimental head for the same flow. The series pump performance had some agreement between the theoretical and experimental data. The theoretical head was lower than the experimental head for the same flow.

The head and flow coefficients for the geometrically similar pumps fell only somewhat collapse onto the same curve. The head and flow coefficients for the geometrically dissimilar pumps fell more so onto the same curve. This suggests that the pumps are geometrically dissimilar, which is consistent with the manufacturer's specifications.

The pump efficiencies were calculated for the experimental and manufacturer data. The pump was most efficient when operating at 2700 $\unit{\rpm}$ in a partially closed valve configuration. The pump was least efficient when operating at 1800 $\unit{\rpm}$ in a partially closed valve configuration. The actual pump efficiency was lower than the manufacturer pump efficiency for all pump speeds.

The vertical elevation of the pump lines have an effect on the head of the pump. This variation was not accounted for in the modelling. It was assumed that this variation was negligible.

\subsection{Technical Recommendations}
This investigation revealed how pump performance can vary significantly from the manufacturer's specifications. It is critical to test the pump before designing a pump system, as the manufacturer's specifications may not be accurate. Due to performance issues and unknown history, replacing or repairing the pump is recommended. 

Further work in improving flow rate measurements through utilizing a calibrated digital flow meter should be considered to reduce the uncertainty in the results.
The largest source of error for all calculated parameters was the precision error from time. This was due to the reaction time of the stopwatch operator. To reduce this error, a digital flow meter should be used to measure the flow rate. This would reduce the error in the flow rate and the head.

The pressure transducer was assumed to have no precision error since calibration was not performed and only one measurement was taken. To account for this error, the pressure transducer should be calibrated and multiple measurements should be taken.
