% Pumps, turbines and fans are machines whose function is to change the energy level of a
% fluid. A pump or compressor increases the total head or pressure of the fluid while a turbine
% decreases it and extracts energy from the flow.
% Turbo machines are widely used. Some examples include
% • an aircraft jet engine has one or more compressors and turbines,
% • electricity is generated by passing steam or other gases through a turbine,
% • liquids are moved long distances and to higher elevations by different types of pumps,
% • air-conditioning and heating systems use fans to circulate air,
% • water wheels and windmills are forms of the turbine, which extract power from naturally
% moving flows.
% The geometry of turbo machines varies appreciably for the differing types that have been
% developed. Broadly there are two classes. In the first class there is a pronounced change in
% radius from the inlet to the discharge; these may be said to be centrifugal turbo machines.
% This is an important type of turbo machine as there are a great number of pumps, turbines
% and compressors that fall into this category. The other class consists of axial machines in
% which the flow is largely parallel to the axis of rotation. Between these extremes are examples
% in which the flow may proceed along conical surfaces of revolution, and these are sometimes
% called mixed-flow turbo machines. In all these varying types, however, there must be a
% rotating member, usually called a rotor, or impeller, to do work on the fluid.
% Turbo machines are an important piece of technology, and a knowledge of their general
% characteristics is essential for many branches of engineering. A study of their operation
% provides an excellent example of the application of fluid mechanics to engineering problems.
% 2 Objectives
% • To measure the performance of a centrifugal pump and compare the results with the
% manufacturer’s specification and theoretical predictions.
% • To compare the performance of parallel and series pump system configurations.

\section{Introduction}
\subsection{Background}
\textbf{TO DO: Reword and find sources for information.}
Pumps, turbines, and fans are turbo machines whose function is to change the energy level of a fluid. A pump or compressor increases the total head or pressure of the fluid while a turbine decreases it and extracts energy from the flow. Analysis of turbo machines is an important piece of technology, and a knowledge of their general characteristics is essential for many branches of engineering. 

The geometry of turbo machines varies appreciably for the differing types that have been developed. Broadly there are two classes. In the first class there is a pronounced change in radius from the inlet to the discharge; these may be said to be centrifugal turbo machines. This is an important type of turbo machine as there are a great number of pumps, turbines and compressors that fall into this category. The other class consists of axial machines in which the flow is largely parallel to the axis of rotation. Between these extremes are examples in which the flow may proceed along conical surfaces of revolution, and these are sometimes called mixed-flow turbo machines. In all these varying types, however, there must be a rotating member, usually called a rotor, or impeller, to do work on the fluid.

\subsection{Objectives}
\begin{enumerate}
    \item To measure the performance of a centrifugal pump and compare the results with the manufacturer’s specification and theoretical predictions.
    \item To compare the performance of parallel and series pump system configurations.
\end{enumerate}

% test table
\begin{table}[h]
    \centering
    \begin{tabular}{|c|c|c|c|c|}
    \hline
    \textbf{Pump} & \textbf{Flow Rate (L/min)} & \textbf{Head (m)} & \textbf{Power (W)} & \textbf{Efficiency (\%)} \\ \hline
    1            & 0asdasdasd                          & 0                & 0                  & 0                       \\ \hline
    2            & 0                          & 0                & 0                  & 0                       \\ \hline
    3            & 0                          & 0                & 0                  & 0                       \\ \hline
    \end{tabular}
    \caption{Pump Performance}
    \label{tab:pump_performance}
\end{table}
