% Apparatus
% • Pump system that can be configured as a single pump or as two pumps in series or
% parallel operation.
% Mec E 403 2
% Centrifugal Pumps
% • Stop watch
% • Strobotach
% • Pressure transducers

% Measurements of pump pressure differential vs. flow rate are required for various operating
% speeds. The manufacturer’s specifications are given in the attached performance curves. The
% impeller has a diameter of 108 mm as shown in Figure 3.
% The pump pressure differential is measured using a calibrated pressure transducer connected
% to pressure taps located upstream of the pump suction inlet and downstream from the outlet.
% Mec E 403 8
% Centrifugal Pumps
% Figure 5: Two pumps operating in parallel
% The suction line has a diameter of 50.8 mm and the discharge line has a diameter of 38.1 mm.
% The flow rate is determined by diverting the flow to a weighing tank and using a stopwatch
% to collect a known quantity of water.
% 1. One pump should be tested individually at 3 speeds specified by the teaching assistant,
% ranging from one-half to full speed. The speed of the pump will be determined using
% the strobotach.
% 2. At each pump speed, the flow rates at 4 different pressures should be recorded (shutoff,
% full open and two other equally spaced intermediate settings). Make sure to adjust the
% speed of the pump to the correct value when changing the pressures before taking any
% other readings. At each operating condition you should record
% a) pump speed (make sure it is set to the correct value),
% b) pressure transducer output,
% c) the moment arm and dyno mass (to determine the torque produced by the motor)
% d) time required to collect a known quantity of water.
% 3. This procedure will then be repeated for parallel and series system configurations with
% both pumps set at full speed (or as specified by the teaching assistant).

\section{Procedure}
\begin{figure}[h]
    \centering
    \includegraphics[width=0.7\textwidth]{Sections/Figures/Experimental Setup.jpg}
    \label{fig:experimental_setup}
    \caption{Experimental setup for the pump system}
\end{figure}
\subsection{Equipment}
\begin{itemize}
    \item Pump system that can be configured as a single pump or as two pumps in series or parallel operation. This is the system being studied.
    \item Stopwatch to measure time to fill a container 
    \item Strobotach to verify pump rotational speed
    \item Mass scale to measure 200 lbs of water
    \item 1.401 kg mass to measure moment arm (torque) of motor
    \item Pressure transducers to measure pump pressure differential
\end{itemize}

\subsection{Procedure}
\begin{enumerate}[label=\arabic*.]
    \item The pump system was tested in single pump configuration at 1800, 2700, and 3600 RPM. The speed of the pump was verified using the strobotach, once, after changing the pump speed.
    \item At each pump speed (1800, 2700, and 3600 RPM), the following measurements at four different pressures were recorded (closed, full open, and two other equally spaced intermediate settings):
    \begin{enumerate}[label=\roman*.]
        \item Time to fill a container to 200 lb of water was recorded three times per given speed and pressure setting. After opening valve 1 (Fig.\ref{fig:experimental_setup}), the "200 lb" weight was placed on the measuring beam. As the mass of the container increased, the beam was raised. The time was recorded when the beam was level. Then, the water container was cleared with the sump pump. 
        
        \indent A total of 36 time measurements were recorded for the single pump configuration, 12 measurements for the parallel, and 12 measurements for the series configurations.
        \item Pressure transducers voltage output was recorded. For the single pump configuration, only the output of the transducer for the active pump was recorded. 
        
        \indent The transducers were measured once per given speed and pressure setting. A total of 12 pressure transducer measurements were recorded for the single pump configuration, 8 measurements for the parallel, and 8 measurements for the series configurations. 
        \item The moment arm were recorded to determine the torque produced by the motor. The dynomass was balanced along the beam attached to the pump until level. The distance of the dyno mass was recorded.
        
        \indent The moment arm was recorded once per given speed and pressure setting. The dyno mass was recorded once per given speed. A total of 12 moment arm and dyno mass measurements were recorded. The moment arm was neglected for the parallel and series configurations.
    \end{enumerate}
    \item Item (2) was repeated for parallel and series system configurations with both pumps set at 2700 RPM (neglect the 1800 and 3600 RPM configurations for parallel and series systems). The moment arm was neglected and the output of both pressure transducers were recorded.
\end{enumerate}