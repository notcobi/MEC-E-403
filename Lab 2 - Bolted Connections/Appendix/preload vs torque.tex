\section{Appendix: Preload-Torque Test Data Analysis}
\label{app:preload_vs_torque_analysis}
The following is the analysis of the preload-torque test data. The data was collected from the experiment and is shown in Table \ref{tab:preload_vs_torque}. The data was then analyzed to determine the preload, preload uncertainty, and torque coefficient. The following sections will detail the analysis of the data and the results of the analysis.

\subsection{Preload vs Torque Analysis}
The results from the experiment are shown in Table \ref{tab:preload_vs_torque}. Sample calculations will be shown for the second row of the table.
\begin{table}[h]
    \centering
    \caption{Torque-Preload Test at Zero External Load}
    \label{tab:preload_vs_torque}
    \begin{tabular}{ccC{0.12\textwidth}C{0.12\textwidth}C{0.12\textwidth}C{0.12\textwidth}C{0.12\textwidth}}
    \toprule
    Torque, $T$ & Torque, $T$ & Bolt Transducer, $V_b$ & Washer Transducer, $V_w$ & Bolt Strain, $\varepsilon_b$ & Preload, $F_i$ & Preload Uncertainty, $\delta F_i$ \\
    (in-lb) & (Nm) & (V) & (V) & & (kN) & ($\pm$ kN) \\
    \midrule
    0 & 0 & -0.002 & -0.001 & -2.00E-06 & -0.0236 & 0.366 \\
    25 & 2.825 & 0.140 & -0.311 & 1.40E-04 & 1.65 & 0.368 \\
    50 & 5.649 & 0.294 & -0.615 & 2.94E-04 & 3.47 & 0.375 \\
    75 & 8.474 & 0.447 & -0.907 & 4.47E-04 & 5.28 & 0.386 \\
    100 & 11.298 & 0.602 & -1.203 & 6.02E-04 & 7.11 & 0.401 \\
    125 & 14.123 & 0.778 & -1.519 & 7.78E-04 & 9.18 & 0.422 \\
    \bottomrule
    \end{tabular}
\end{table}
First, the torque was converted to metric units.
\begin{align*}
    T &= \qty{25}{\inlb} \times \qty{0.113}{\newton\per\meter\per\inlb} \\
    &= \qty{2.825}{\newton\meter}
\end{align*}
The bolt strain, $\varepsilon_b$, was then calculated by
\begin{align*}
    \varepsilon_b &= \frac{4 V_b}{K_g E_{\text{in}} G} \\
    &= \frac{4 \times \qty{0.140}{V}}{2 \times \qty{5}{V} \times 400} \\
    &= \qty{1.40E-04}{}
\end{align*}
The preload, $F_i$, was then calculated by
\begin{align*}
    F_i &= E_b \varepsilon_b A_1 \\
    &= \qty{205.046}{\giga\pascal} \times \qty{1.40E-04} \times \qty{57.570}{\milli\meter\squared} \\
    &= \boxed{\qty{1.65}{\kilo\newton}}
\end{align*}

\subsection{Uncertainty Analysis of Preload}
A repeatability test was performed at 50 lb-in of torque with no external load. The results of this test are shown in 
\begin{table}[h]
    \centering
    \caption{Repeatability Test at 50 lb-in of Torque and Zero External Load}
    \label{tab:repeatability_test}
    \begin{tabular}{ccc}
    \toprule
    Trial \# & Bolt Transducer, $V_b$ & Washer Transducer \\
    & (V) & (V) \\
    \midrule
    1 & 0.372 & -0.701 \\
    2 & 0.321 & -0.684 \\
    3 & 0.354 & -0.718 \\
    4 & 0.312 & -0.654 \\
    5 & 0.327 & -0.679 \\
    \bottomrule
    \end{tabular}
\end{table}
The standard deviation was determined with Excel to be $S_{V_b} = \qty{0.0250}{\volt}$. Using a confidence level of 95\%, the t-distribution value was determined by
\begin{align*}
    \alpha/2 &= \frac{1 - 0.95}{2} = 0.025 \\
    n - 1 &= 5 - 1 = 4 \\
    t_{\alpha/2, n-1} &= 2.776
\end{align*}
The precision uncertainty is then
\begin{align*}
    P_{V_b} &= t_{\alpha/2, n-1} \cdot \frac{S_{V_b}}{\sqrt{n}} \\
    &= 2.776 \cdot \frac{\qty{0.025}{\volt}}{\sqrt{5}} \\
    &= \qty{0.031}{\volt}
\end{align*}
Defining bias uncertainty as resolution, $B_{V_b} = 0.001$, the total uncertainty is then
\begin{align*}
    \delta V_b &= \sqrt{P_{V_b}^2 + B_{V_b}^2} \\
    &= \sqrt{(\qty{0.031}{\volt})^2 + (\qty{0.001}{\volt})^2} \\
    &= \qty{0.031}{\volt}
\end{align*}
The uncertainty of the preload for the second row of Table \ref{tab:preload_vs_torque} is then
\begin{align*}
    \delta F_i &= F_i \sqrt{\left(\frac{\delta V_b}{V_b}\right)^2 + \left(\frac{\delta E_b}{E_b}\right)^2} \\
    &= \qty{1.65}{\kilo\newton} \sqrt{\left(\frac{\qty{0.031}{\volt}}{\qty{0.140}{\volt}}\right)^2 + \left(\frac{\qty{4.70}{\giga\pascal}}{\qty{205.046}{\giga\pascal}}\right)^2} \\
    &= \qty{0.368}{\kilo\newton}
\end{align*}

\subsection{Torque Coefficient Analysis}
Applyig linear regression, forced through the origin, to the data in Table \ref{tab:preload_vs_torque} using \texttt{=LINEST()} from Excel, the equation is,
\begin{align*}
%    0.635762559
    F_i &= 0.636 T \\
    \implies \frac{T}{F_i} &= \frac{1}{0.636} \unit{\per\milli\meter}
\end{align*}
where $F_i$ is in kN and $T$ is in Nm. From Eq. (\ref{eq:torque_constant}), the torque coefficient is then
\begin{align*}
    K &= \frac{T}{F_i d} \\
    &= \frac{1}{\qty{0.636}{\per\milli\meter} \times \qty{0.375}{\inch} \times \qty{25.4}{\milli\meter\per\inch}} \\
    &= \boxed{\qty{0.167}{}}
\end{align*}


