\section{Appendix: Zero Preload Data Analysis}
\label{app:zero_preload_analysis}
This Appendix provides the analysis of the experimental data "bolt stiffness and washer calibration (finger tight)" to determine the modulus of elasticity of the bolt. In addition, error analysis was performed with a confidence of 95\% to determine the corresponding uncertainty. In addition, the washer calibration was also performed to determine the relationship between the external load, washer transducer reading, and washer strain.

\subsection{Modulus of Elasticity Analysis}
\begin{table}[h]
    \centering
    \caption{Bolt Stiffness and Washer Calibration data}
    \label{tab:modulus_of_elasticity_data}
    \begin{tabular}{cccc}
    \toprule
    External Load, $P$ & Bolt Out, $V_{o, b}$ & Washer Out, $V_{o, w}$ & Bolt Strain, $\varepsilon_b$ \\
    (kN) & (V) & (V) & \\
    \midrule
    0 & 0.006 & -0.003 & 6.00E-06 \\
    1 & 0.085 & -0.212 & 8.50E-05 \\
    2 & 0.168 & -0.441 & 1.68E-04 \\
    3 & 0.253 & -0.667 & 2.53E-04 \\
    4 & 0.335 & -0.888 & 3.35E-04 \\
    5 & 0.417 & -1.08 & 4.17E-04 \\
    6 & 0.498 & -1.27 & 4.98E-04 \\
    7 & 0.581 & -1.46 & 5.81E-04 \\
    7.5 & 0.662 & -1.547 & 6.62E-04 \\
    \bottomrule
    \end{tabular}
\end{table}
The experimental data was collected and shown in Table \ref{tab:modulus_of_elasticity_data}. Sample calculations will be shown for external load of 0 kN. The bolt strain was calculated from Eq. (\ref{eq:strain_bridge}),
\begin{align*}
    \varepsilon &= \frac{4 V_{o, b}}{K_g E_{\text{in}} G} \\
    \varepsilon &= \frac{4 \cdot \qty{0.006}{V}}{2 \cdot \qty{5}{V} \cdot 400} \\
    &= 6.00 \times 10^{-6}
\end{align*}
where $E_o$ is transducer reading, $K_g$ is the gauge factor, $E_{\text{in}}$ is the voltage input, and $G$ is the gain set. From the experimental setup, $K_g = 2$, $E_{\text{in}} = 5$, and $G = 400$. 

Next, a linear regression of the external load ($P$) and bolt strain ($\varepsilon_b$), forced through the origin, was performed on the data in Table \ref{tab:modulus_of_elasticity_data} to determine the modulus of elasticity. The linear regression equation was determined using \texttt{=LINEST()} from Excel. The results are shown in Table \ref{tab:modulus_of_elasticity_regression}. The equation is then
\begin{align*}
    \varepsilon_b &= 8.47134 \times 10^{-5} P
\end{align*}
or in another form,
\begin{align*}
    \frac{P}{\varepsilon_b} &= \frac{1}{8.47134 \times 10^{-5}} 
\end{align*}
\begin{table}[h]
    \centering
    \caption{Linear Regression Results}
    \label{tab:modulus_of_elasticity_regression}
    \begin{tabular}{lc}
    \toprule
    Parameter & Value \\
    \midrule
    Slope (mm/kN) & 8.47134E-05 \\
    Slope Standard Error, $S_a$ & 8.20567E-07 \\
    $R^2$ & 0.999249954 \\
    \bottomrule
    \end{tabular}
\end{table}
The area where the force was applied is the outer diameter, $d_o$, minus the inner diameter, $d_i$, of the bolt. From the experimental setup, $d_o = \qty{0.371}{\inch}$ and $d_i = \qty{0.155}{\inch}$. The area is then
\begin{align*}
    A_1 &= \frac{\pi}{4} (d_o^2 - d_i^2) \\
    &= \frac{\pi}{4} ((\qty{0.371}{\inch}\times \qty{25.4}{\milli\meter\per\inch})^2 - (\qty{0.155}{\inch}\times \qty{25.4}{\milli\meter\per\inch})^2) \\
    &= \qty{57.570}{\milli\meter\squared}
\end{align*}
The modulus of elasticity is then
\begin{align*}
    E &= \frac{P}{\varepsilon_b A_1} \\
    &= \frac{\qty{1}{\kilo\newton}}{8.47134 \times 10^{-5} \times \qty{57.570}{\milli\meter\squared}} \\
    &= \boxed{\qty{205}{\giga\pascal}}
\end{align*}   

\subsection{Modulus of Elasticity Error Analysis}
\subsubsection{Error Propagation Derivation}
\label{sec:error_propagation_derivation}
To be thorough, the error propagation formula will be derived for completeness. 

If we know how a quantity of interest depends on other, directly measurable quantizes, it is posable to estimate the uncertainty of this "output" quantity based on the uncertainties in the measured quantizes. For example, we can calculate the uncertainty associated to a volume based on the uncertainty of the measurement of the individual dimensions.

Consider as results, $R$, which is a function of $n$ variables, $x_1, \ldots, x_n$:
\begin{align*}
    R &= f(x_1, \ldots, x_n)
\end{align*}
If the individual measurands, $x_i$, have an associated uncertainty $w_{x_i}$, what is teh uncertainty of $w_R$ of the result R?

Defining $x:= (x_1, \ldots, x_n)$, and $x_m := (x_{m, 1}, \ldots, x_{m, n})$, perform the Taylor series expansion of $R= f(x)$ about the point $x = x_m$, taking $x_i - x_{m, i} = w_{x_i}$:
\begin{gather*}
    R = f(x_m) + \frac{\partial f}{\partial x_1} \bigg|_{x = x_m} \underbrace{(x_1 - x_{m, 1})}_{w_{x_1}} + \ldots + \frac{\partial f}{\partial x_n} \bigg|_{x = x_m} \underbrace{(x_n - x_{m, n})}_{w_{x_n}} + \text{H.O.T.} \\
    \underbrace{R - f(x_m)}_{w_R} = \frac{\partial f}{\partial x_1} \bigg|_{x = x_m} w_{x_1} + \ldots + \frac{\partial f}{\partial x_n} \bigg|_{x = x_m} w_{x_n} + \text{H.O.T.}
\end{gather*}
The higher-order terms contain quadratic terms $w_{x_i}w_{x_j}$, cubic terms $w_{x_i}w_{x_j}w_{x_k}$, and so on. Assuming the individual uncertainties $w_{x_i}$ are small,  we can take these higher-order terms as zero, giving 
\begin{align*}
    w_R &= \sum_{i=1}^n \bigg|\frac{\partial f}{\partial x_i} \bigg|_{x = x_m} w_{x_i} \bigg|
\end{align*}
However, this is the worst-case uncertainty, and is an overly conservative estimate. A better estimate is to use the root of sum of squares
\begin{align}
    w_R &= \sqrt{\sum_{i=1}^n \left[\frac{\partial f}{\partial x_i} \bigg|_{x = x_m} w_{x_i}\right]^2} \label{eq:error_propagation}
\end{align}
If the confidence levels associated to the individual uncertainties $w_{x_i}$ are all identical (for instance 95\%), the confidence level of the result $w_R$ will be the same. 

The key assumption behind RSS is that the set of measured variables $x_1, \ldots, x_n$ are \textbf{statistically indecent}. If this is not the case, a different formulation needs to be used. Also note that all uncertainties $w_{x_i}$ need to be small such that the first-order Taylor series approximation holds. 

Consider the case where the result $R$ is dependent only on the product of the measured variables, $x_1, \ldots, x_n$ with associated uncertainties $w_{x_1}, \ldots, w_{x_n}$ as 
\begin{align*}
    R = C x_{1}^{c_1} x_{2}^{c_2} \ldots x_{n}^{c_n}
\end{align*}
where $C$ and $c_1, \ldots, c_n$ are constants. In this case, the RSS formula gives
\begin{align}
    w_{R} = \sqrt{\left(C c_1 x_{1}^{c_1 - 1} w_{x_1}\right)^2 + \ldots + \left(C c_n x_{1}^{c_1} x_{2}^{c_2} \ldots c_n x_{n}^{c_n - 1} w_{x_n}\right)^2} \nonumber \\
    \implies \frac{w_{R}}{|R|} = \sqrt{\left(\frac{c_1 w_{x_1}}{x_1}\right)^2 + \ldots + \left(\frac{c_n w_{x_n}}{x_n}\right)^2} \label{eq:error_propagation_pure_multiplicative}
\end{align}

\subsubsection{Modulus of Elasticity Error Analysis}
The uncertainty of slope was determined using the standard error of the slope, $S_a$, from the linear regression in Table \ref{tab:modulus_of_elasticity_regression} at a confidence level of 95\%. The t-distribution value was determined by 
\begin{align*}
    \alpha/2 &= \frac{1 - 0.95}{2} = 0.025 \\
    n - 2 &= 9 - 2 = 7 \\
    t_{\alpha/2, n-2} &= 2.3646
\end{align*}
Recall from MEC E 300 that the error for the coefficients for the form $f(x) = ax +b$ can be found by finding standard error of $a$ and $b$ by first finding 
\begin{align*}
    S_{y,x} &= \sqrt{\frac{\sum (y_i - a x_i - b)^2}{n-2}} \\
\end{align*}
then,
\begin{align*}
    S_a &= S_{y,x} \sqrt{\frac{1}{\sum (x_i - \bar{x})^2}} \\
    S_b &= S_{y,x} \sqrt{\frac{\sum x_i^2}{n \sum (x_i - \bar{x})^2}}
\end{align*}
then with a given confidence of $1 - \alpha$, the uncertainties are 
\begin{align*}
    a \pm t_{\alpha/2, n-2} S_a \\
    b \pm t_{\alpha/2, n-2} S_b
\end{align*}
The uncertainty of the slope is then \cite{wheeler_ganji}
\begin{align*}
    \delta \text{slope} &= t_{\alpha/2, n-2} \cdot S_a \\
    &= 2.3646 \cdot 8.20567 \times 10^{-7} \\
    &= \qty{1.94E-06}{\per\kilo\newton}
\end{align*}
The function for modulus of elasticity is 
\begin{align*}
    E &= P^{1} \varepsilon_b^{-1} A_1^{-1} \\
    &= (\text{slope})^{-1} A_1^{-1}
\end{align*}
This is the purely multiplicative case for error propagation \cite{wheeler_ganji}.  Which is, from Eq. (\ref{eq:error_propagation_pure_multiplicative}),
\begin{align*}
    \frac{\delta E}{|E|} &= \sqrt{\left(\frac{1}{\text{slope}} \delta \text{slope}\right)^2 + \left(\frac{1}{A_1} \delta A_1\right)^2}
\end{align*}
Assuming the error for $A_1$ is negligible, then by Eq. (\ref{eq:error_propagation_pure_multiplicative}), the uncertainty of the modulus of elasticity is then
\begin{align*}
    \delta E &= E \bigg|\frac{\delta \text{slope}}{\text{slope}}\bigg|\\
    &= \qty{205}{\giga\pascal} \frac{\qty{1.94E-06}{\per\kilo\newton}}{\qty{8.47134E-05}{\per\kilo\newton}} \\
    &= \boxed{\pm \qty{4.70}{\giga\pascal}}
\end{align*}
This quantity is relatively small compared to the modulus of elasticity, and is expected. Please, let me know if I need to show an even more formal statistical derivation of error propagation to satisfy any pedantic needs.

\subsection{Washer Calibration Analysis}
The external load and washer transducer readings from Table \ref{tab:modulus_of_elasticity_data} were fitted with a linear regression through the origin. The linear regression equation was determined using \texttt{=LINEST()} from Excel. The equation was 
\begin{align*}
    \Aboxed{E_{o, w} &= -0.211 P}
\end{align*}
