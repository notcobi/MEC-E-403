%\begin{abstract}
\section*{Abstract}
    Bolted connections are integral components in various engineering structures, facilitating disassembly without damage and offering versatility in construction. Understanding their mechanics is paramount for ensuring structural integrity and reliability in engineering applications. This study delves into the intricacies of bolted connections, examining key parameters such as modulus of elasticity, preload, stress and strain, washer calibration, torque coefficient, bolt stiffness, and member stiffness. Through experimentation and analysis, this research seeks to understand the mechanics of bolted connections, providing groundwork for future reliable engineering practices.

    The experimental investigation involved utilizing an MTS testing machine and instrumented bolted joints, including strain conditioner and gauges, and a torque wrench. Six tests were conducted: zero preload, zero load, repeated loading, static loading, shake down, and dynamic loading. 

    Results from the zero preload trial indicate a modulus of elasticity of $205 \pm 4.70$ GPa, a bolt calibration of $\varepsilon_{b} = 8.47E-05 P$, preload uncertainty of $\pm 5\%$ kN, and washer calibration of $-2.11 \times 10^{-1}P$. 
    
    Results from the zero load trial found torque coefficient, $K = 0.167$, which falls within the expected range of $0.1 - 0.2$. 
    
    The uncertainty for the bolt transducer reading was determined from a repeatability test to be $\delta V_{o, b} =\pm 0.03$
    
    Next, assuming the stress distribution was $45^{\circ}$, the theoretical stiffness of the joined members was determined to be $k_{m, th} = 2222$ MN/m. The static loading trials found the experimental stiffness of the joined members to be $k_{m, exp} = 1600$ MN/m. The relative error between the theoretical and experimental  stiffness of the member was $28.0\%$. 

    From the static loading trials, the separation point for the bolt without a gasket was found to be $P = 4.98$ kN experimentally and $P = 4.67$ kN theoretically, with a $6.78\%$ relative error. It was observed that the gasket decreased the bolt force for a given external load, and the gasket helped prevent the members from separating.

    The study also found that both the mean and alternating stresses increased as torque increased. The gasket generally increased the mean stress and alternating stress. The alternating stress decreased as torque increased, and the mean stress increased as torque increased.

    This report gave insight to the loading patterns experienced by bolts and the resulting changes induced by varying parameters on connection stress. This may serve as a reference of understanding to compliment the theory behind bolted connections, the fundamentals of strength testing, diverse loading methodologies, the significance of preload, disparities between members with and without a gasket, and the interrelationships among pertinent parameters.
%\end{abstract}