\section{Conclusion}
The modulus of elasticity was determined to be $E = 205 \pm 4.70$ GPa. This quantity was determined from the zero preload trial. The regression used to determine the modulus was $\varepsilon_{b} = 8.47E-05 P$ and had an $R^2 = 0.9992$, indicating a strong linear relationship between the external load and the bolt strain. The preload uncertainty was determined to be $\pm 5\%$ kN from the zero loading trial. The uncertainty is relatively small, which was calculated from the standard error $S_{a}$ of the regression.

The washer calibration was found to be $V_{o, w} = -2.11 \times 10^{-4}P$. This was also calculated from the zero preload trial. This allows prediction of the washer voltage output for a given external load. The regression had an $R^2 = 0.9994$, indicating a strong linear relationship between the external load and the washer transducer.

The torque coefficient was determined to be $0.167$. This was from the zero loading trial, where the torque was varied and the bolt strain was measured. The regression had an $R^2 = 0.9995$, indicating a strong linear relationship between the torque and the preload. The slope value was then used to determine the torque coefficient. This value was within the expected range of $0.1 - 0.2$, adding confidence to the results.

A shakedown test was performed to determine if the bolt was subjected to any torsional loading. The voltage output varied by $\pm 0.005$ V, which was small, the same magnitude as the resolution of the strain transducer. This suggest that the bolt was not subjected to any torsional loading. 

The uncertainty for the bolt transducer reading was determined from a repeatability test to be $\delta V_{o, b} =\pm 0.03$. This was moderate, being a magnitude larger than the resolution of the strain transducer. 
The stiffness of the bolt was determined to be $k_b = 209$ MN/m. The bolt was divided into three sections, and the stiffness of each section was determined by modelling the bolt as three distinct sections which act as three springs in series, with stiffness $k_1 = 510$ MN/m, $k_2 = 383$ MN/m, and $k_3 = 4834$ MN/m. 

% don't reference sections
The theoretical and experimental stiffness of the joined members was determined to be $k_m = 2222$ MN/m and $k_m = 1600$ MN/m, respectively. The theoretical stiffness of the joined members was determined by modelling the angle of stress distribution on the member to be 45$^\circ$. The experimental stiffness of the member was determined using the preseparation regression from the static loading no gasket trial. The constant $C$ was determined to be $0.1157$ from the preseparation regression ($R^2 = 0.9983$). The relative error between the theoretical and experimental stiffness of the member was $28.0\%$. The assumption of the stress distribution angle was likely responsible for discrepancies between the theoretical and experimental stiffness of the member. 

During the static loading trials, the gasket decreased the bolt force for a given external load. A separation was observed in the no gasket trial, but not in the gasket trial. This suggest the gasket helped prevent the members from separating. Further testing could be done to determine the separation point of the gasket. The $R^2$ values for the no gasket trial and gasket trial were $0.9983$ and $0.9985$, respectively, indicating a strong linear and quadratic relationship between the external load and the bolt strain. 

The separation point was observed to be $P = 4.98$ kN and $P = 4.67$ kN for the experimental and theoretical values, respectively. The theoretical separation point was determined using the experimental $k_b$ and theoretical $k_m$, whereas the experimental separation point was determined by equating the preseparation and postseparation regressions from the no gasket trial. The relative error between the experimental and theoretical separation points was $6.78\%$. The main discrepancy came from the difference in the experimental and theoretical stiffness of the member, which had a relative error of $28.0\%$.

Both the mean and alternating stresses increased as torque increased. The gasket generally increased the mean stress and alternating stress. The alternating stress decreased as torque increased, and the mean stress increased as torque increased.

The objectives of the lab were achieved: the modulus of elasticity was obtained, the stiffness of the member and bolts were determined, and insights into the effects of torque, preload, external loads, gaskets, and dynamic loading were obtained. The results were consistent with expectations, and the uncertainty was relatively small. The largest sources of error was the uncertainty from the strain transducer, $V_{o, b}$ and the modulus, $E_b$. Future work could be done to verify the stress distribution angle, and to determine the separation point of the gasket.

Understanding these parameters will aid in design and analysis of bolted connections in critical applications. Ensuring safe and reliable operation of bolted connections is essential in many engineering applications, and the results of this lab will be useful in future work.

\section{Technical Recommendations}
One transducer reading for the washer and bolt was measured for a given external load for the zero preload trial. This totaled to nine measurements for the linear regression. Future work could expand by taking three measurements for the washer and bolt transducer for a given external load, which should reduce the standard error of the regression. More thorough calibration of the washer and bolt transducer could be done to reduce the bias uncertainty.

In addition, hysteresis was not accounted for in the zero preload trial, as measurements were only taken upwards. Performing the measurements in both directions could be used to determine the effect of hysteresis on the modulus of elasticity. These recommendations could be used to increase the accuracy and confidence of the modulus of elasticity.

The theoretical value for the stiffness of the member did not match the experimental value, and the effects rippled through the theoretical separation point calculations. The stress distribution angle was assumed to be 45$^\circ$, and was never verified. Future work could be done to verify the stress distribution angle, which could be used to determine the stiffness of the member with higher accuracy and confidence.

The gasket was not tested for separation, and the separation point was not determined. Knowing this could be a critical design parameter. For example, in a boiler, the gasket could be subjected to high temperatures and pressures, and the separation point could be critical to ensure the gasket does not fail. Future work could be done to determine the separation point of the gasket.

The torque wrench was not calibrated, and the uncertainty of the torque wrench was not determined. The torque wrench may be inaccurate, as the operator can go past the click point. Utilizing a digital solution could be more accurate to apply the torque on the bolt.