\section{Introduction}
Bolted connections are frequently utilized in engineering to secure rigid members, serving as a crucial element in various structures. Although they may seem straightforward, their importance cannot be emphasized enough. Ensuring the strength of these connections through testing is essential to prevent failure under load. Neglecting proper criteria can lead to severe consequences, as evidenced by the incident involving the Boeing 737 Max. In this case, a loose bolt was discovered in the rudder-control system during routine maintenance, averting a potential catastrophic incident \cite{levenson_boeing_2023}.

The primary focus of this report is to analyze different loading conditions on bolted members and investigate how specific factors influence the strength of the joined members. The laboratory procedure utilizes an MTS testing machine to measure the stress and strain experienced by the bolt. Static load testing is conducted on the connection to determine Young's Modulus. Additionally, the relationship between torque and preload is experimentally determined. A shakedown test is performed to assess whether the bolt experiences torsional loading, and the stiffness of the bolt is determined using the experimentally derived Young's Modulus. The stiffness of the assembled member is evaluated both theoretically and experimentally during static loading trials. These trials also analyze the effect of preload and draw conclusions regarding joint separation. Furthermore, static tests are conducted both with and without a gasket in the connection to study the impact of the gasket on the member. A dynamic loading trial is also carried out, focusing on the influence of preload and gasket contribution.

This report provides insights into the loading patterns experienced by bolts and the resulting changes induced by varying parameters on connection stress. It aims to serve as a comprehensive reference elucidating the theory behind bolted connections, the fundamentals of strength testing, diverse loading methodologies, the significance of preload, disparities between members with and without a gasket, and the interrelationships among pertinent parameters.