\section{Appendix: Damping Ratio}
\label{app:Damping Ratio Calculation}
% From the free vibration chart traces, select several pairs of maxima along the decaying
% oscillation and calculate an average damping ratio, ζ. Comment on how well or poorly
% the linear viscous damping model approximates the energy dissipation in the air track
% system. Justify your answer.


\subsection{Damping Ratio Calculation}
\begin{longtable}{ccccc}
    \caption{Damping Ratio Data}
    \label{tab:damping_ratio} \\
    \toprule
    Dataset & Peak Number & Amplitude, $x$ & $\delta$ & $\zeta$ \\
    & & (cm) & & \\
    \midrule
    \endfirsthead
    \toprule
    Dataset & Peak Number & Amplitude, $x$ & $\delta$ & $\zeta$ \\
    & & (cm) & & \\
    \midrule
    \endhead
    \bottomrule
    \endfoot
    Big 1 & 1 & 8.2775 & 0.119385043 & 0.018997291 \\
    Big 1 & 2 & 7.346 & 0.134348497 & 0.021377341 \\
    Big 1 & 3 & 6.4225 & 0.146730503 & 0.02334652 \\
    Big 1 & 4 & 5.546 & 0.179764425 & 0.028598694 \\
    Big 1 & 5 & 4.6335 & 0.202002469 & 0.032133089 \\
    Big 1 & 6 & 3.786 & 0.240058119 & 0.038178581 \\
    Big 1 & 7 & 2.978 & 0.328811927 & 0.052260531 \\
    Big 1 & 8 & 2.1435 & 0.471264045 & 0.074793917 \\
    Big 1 & 9 & 1.338 & - & - \\
    Big 2 & 1 & 9.4465 & 0.101586118 & 0.01616582 \\
    Big 2 & 2 & 8.534 & 0.116963123 & 0.018612035 \\
    Big 2 & 3 & 7.592 & 0.132252668 & 0.021044005 \\
    Big 2 & 4 & 6.6515 & 0.145034028 & 0.023076735 \\
    Big 2 & 5 & 5.7535 & 0.17320361 & 0.027555743 \\
    Big 2 & 6 & 4.8385 & 0.192437655 & 0.030613049 \\
    Big 2 & 7 & 3.9915 & 0.23835321 & 0.037907825 \\
    Big 2 & 8 & 3.145 & 0.317698849 & 0.05049883 \\
    Big 2 & 9 & 2.289 & 0.444895542 & 0.070630487 \\
    Big 2 & 10 & 1.467 & 0.79495422 & 0.125520247 \\
    Big 2 & 11 & 0.6625 & - & - \\
    Small 1 & 1 & 9.473 & 0.093317023 & 0.014850228 \\
    Small 1 & 2 & 8.629 & 0.102608739 & 0.016328511 \\
    Small 1 & 3 & 7.7875 & 0.118031863 & 0.018782041 \\
    Small 1 & 4 & 6.9205 & 0.131541363 & 0.020930872 \\
    Small 1 & 5 & 6.0675 & 0.147293322 & 0.023436022 \\
    Small 1 & 6 & 5.2365 & 0.185362555 & 0.029488537 \\
    Small 1 & 7 & 4.3505 & 0.211971847 & 0.033717185 \\
    Small 1 & 8 & 3.5195 & 0.27362217 & 0.043507086 \\
    Small 1 & 9 & 2.677 & 0.379561461 & 0.060299159 \\
    Small 1 & 10 & 1.8315 & 0.55872893 & 0.088574955 \\
    Small 1 & 11 & 1.0475 & - & - \\
    Small 2 & 1 & 23.182 & 0.078630971 & 0.012513528 \\
    Small 2 & 2 & 21.429 & 0.091411866 & 0.014547111 \\
    Small 2 & 3 & 19.557 & 0.095540303 & 0.015203954 \\
    Small 2 & 4 & 17.775 & 0.10451699 & 0.016632095 \\
    Small 2 & 5 & 16.011 & 0.1158877 & 0.018440964 \\
    Small 2 & 6 & 14.259 & 0.126950746 & 0.020200716 \\
    Small 2 & 7 & 12.559 & 0.144982969 & 0.023068616 \\
    Small 2 & 8 & 10.864 & 0.178279663 & 0.028362675 \\
    Small 2 & 9 & 9.09 & 0.201245362 & 0.032012778 \\
    Small 2 & 10 & 7.433 & 0.257773714 & 0.040991478 \\
    Small 2 & 11 & 5.744 & 0.350674281 & 0.055724823 \\
    Small 2 & 12 & 4.045 & 0.532063149 & 0.084378491 \\
    Small 2 & 13 & 2.376 & - & - \\
\end{longtable}

The raw data for the damping ratio calculation is shown in Table \ref{tab:damping_ratio}. Sample calculations for the damping ratio are shown for the Big 1 dataset. First, $\delta$ was found by
\begin{align*}
    \delta &= \ln\left(\frac{x_n}{x_{n+1}}\right) \\
    &= \ln\left(\frac{8.2775}{7.346}\right) \\
    &= 0.119385043
\end{align*}
Then, $\zeta$ was found by
\begin{align*}
    \zeta &= \frac{\delta}{\sqrt{4\pi^2 + \delta^2}} \\
    &= \frac{0.119385043}{\sqrt{4\pi^2 + 0.119385043^2}} \\
    &= \boxed{0.018997}
\end{align*}