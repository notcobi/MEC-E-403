\section{Appendix: Natural Frequency}
\label{app:Natural Frequency Derivation}
\subsection{Period Calculation}
\begin{table}[H]
    \centering
    \caption{Natural Frequency Trial Data}
    \label{tab:natural_frequency}
    \begin{tabular}{cccccc}
    \toprule
        %Trial & $f_{\text{no measure}}$ & $f_{\text{measure}}$ & $f_{\text{no measure}}$ & $f_{\text{measure}}$ \\
        & \multicolumn{2}{c}{Small} & \multicolumn{2}{c}{Big} \\
        \cmidrule(lr){2-3} \cmidrule(lr){4-5}
        Trial & 10$\tau_{\text{no sys}}$ & 10$\tau_{\text{sys}}$ & 10$\tau_{\text{no sys}}$ & 10$\tau_{\text{sys}}$ \\
        & (s) & (s) & (s) & (s) \\
        \midrule
        1 & 8.81 & 11.13 & 11.49 & 13.7 \\
        2 & 8.71 & 10.98 & 11.67 & 13.14 \\
        3 & 8.85 & 11.01 & 11.69 & 13.31 \\
        4 & 8.88 & 10.90 & 11.84 & 13.19 \\
        5 & 8.93 & 10.95 & 11.61 & 13.21 \\
        \midrule
        \midrule
        Average & 8.836 & 10.994 & 11.66 & 13.31 \\
        $\tau$ & 0.8836 & 1.0994 & 1.166 & 1.331 \\
        \bottomrule
    \end{tabular}
\end{table}
The experimental data for the period of the system is shown in Table \ref{tab:natural_frequency}. Sample calculations for the period are shown for the small no system trial. The average was found by Excel,
\begin{align*}
    (10\tau_{\text{no sys}})_{\text{avg}} &= \frac{1}{5} \sum_{i=1}^{5} (10\tau_{\text{no sys}})_i \\
    &= \frac{1}{5} \times (8.81 + 8.71 + 8.85 + 8.88 + 8.93) \\
    &= 8.836
\end{align*}
The period was found by
\begin{align*}
    \tau_{\text{no sys}} &= \frac{(10\tau_{\text{no sys}})_{\text{avg}}}{10} \\
    &= \frac{8.836}{10} \\
    &= \boxed{0.8836}
\end{align*}

\subsection{Natural Frequency Calculation}
\begin{table}[H]
    \centering
    \caption{Natural Frequency Data}
    \label{tab:natural_frequency_data}
    \begin{tabular}{cccccc}
    \toprule
        & \multicolumn{2}{c}{Small} & \multicolumn{2}{c}{Big} \\
        \cmidrule(lr){2-3} \cmidrule(lr){4-5}
        & $f_{\text{sys}}$ & $f_{\text{no sys}}$ & $f_{\text{sys}}$ & $f_{\text{no sys}}$ \\
        & (Hz) & (Hz) & (Hz) & (Hz) \\
        \midrule
        Derived from period & 1.1317 & 0.9096 & 0.8576 & 0.7513 \\
        Derived from using $k_e$ and $m_e$ & 1.1355 & 0.9221 & 0.8744 & 0.7651 \\
        \bottomrule
    \end{tabular}
\end{table}

\begin{table}[H]
    \centering
    \caption{Effective Mass Data}
    \label{tab:effective_stiffness_mass}
    \begin{tabular}{ccc}
    \toprule
        & System & No System \\
        & (kg) & (kg) \\
        \midrule
        $m_e$, Small & 0.40155 & 0.2648 \\
        $m_e$, Big & 0.58325 & 0.4465 \\
        \bottomrule
    \end{tabular}
\end{table}
The natural frequency of the system was calculated using the period data in Table \ref{tab:natural_frequency_data}. Sample calculations for the natural frequency are shown for the small system trial. The natural frequency was found by
\begin{align*}
    f_{\text{sys}} &= \frac{1}{\tau_{\text{sys}}} \\
    &= \frac{1}{1.0994} \\
    &= \boxed{0.9096}
\end{align*}
The natural frequency was also calculated using the effective stiffness and mass data. The effective mass data is shown in Table \ref{tab:effective_stiffness_mass}. The effective mass for the system was found by 
\begin{align*}
    m_{e, \text{sys}} &= m_{\text{cart}} + \frac{1}{2} m_{\text{big pulley}} + \frac{1}{2} m_{\text{small pulley}} \\
    &= \qty{0.2648}{\kilo\gram} + \frac{1}{2} \qty{0.2599}{\kilo\gram} + \frac{1}{2} \qty{0.0136}{\kilo\gram} \\
    &= \qty{0.40155}{\kilo\gram}
\end{align*}
The natural frequency was found by 
\begin{align*}
    f_{\text{sys}} &= \frac{1}{2\pi} \sqrt{\frac{k_e}{m_e}} \\
    &= \frac{1}{2\pi} \sqrt{\frac{\qty{13.5}{\newton\per\meter}}{\qty{0.40155}{\kilo\gram}}} \\
    &= \boxed{1.1355}
\end{align*}

\subsection{Natural Frequency Error Analysis}
Important values for uncertainty analysis are presented in Table \ref{tab:natural_frequency_data}.  
\begin{table}[h]
    \centering
    \caption{Natural Frequency Error Analysis}
    \label{tab:natural_frequency_error_analysis}
    \begin{tabular}{cccccc}
    \toprule
        & \multicolumn{2}{c}{Small} & \multicolumn{2}{c}{Big} \\
        \cmidrule(lr){2-3} \cmidrule(lr){4-5}
        & w/ sys & w/o sys & w/ sys & w/o sys \\
        \midrule
        Stdev 10$\tau$ (s) & 0.0829 & 0.0862 & 0.1273 & 0.2266 \\
        t-inv value & 2.7764 & 2.7764 & 2.7764 & 2.7764 \\
        $P_{10\tau}$ (s) & 0.1030 & 0.1070 & 0.1580 & 0.2814 \\
        $B_{10\tau}$ (s)& 0.01 & 0.01 & 0.01 & 0.01 \\
        $\delta_{\tau}$ (s) & 0.0103 & 0.0107 & 0.0158 & 0.0281 \\
        $\delta_p$ (Hz) & 0.013 & 0.009 & 0.012 & 0.016 \\
        \bottomrule
    \end{tabular}
\end{table}
Sample calculations for the error analysis are shown for the small system trial. The standard deviation was calculated by Excel. The precision error for $10\tau$ was found by
\begin{align*}
    P_{10\tau} &= t_{\alpha/2, n-1} \times \frac{\text{stdev } 10\tau}{\sqrt{n}} \\
    &= 2.7764 \times \frac{0.0829}{\sqrt{5}} \\
    &= 0.1030
\end{align*}
where $t_{\alpha/2, n-1} = 2.7764$ for a 95\% confidence interval. The bias error for $\tau$ was set to be the resolution of the stopwatch. Next, the uncertainty for $10\tau$ was found by
\begin{align*}
    \delta_{\tau} &= \sqrt{P_{10\tau}^2 + B_{10\tau}^2} \\
    &= \sqrt{0.1030^2 + 0.01^2} \\
    &= \qty{0.0103}{\second}
\end{align*}
The uncertainty for $\tau$ was found by
\begin{align*}
    \delta_p &= \sqrt{\left(\frac{\partial 10 \tau}{10} \delta_{\tau}\right)^2} \\
    &= \frac{1}{10} \delta_{\tau} \\
    &= \frac{0.0103}{10} \\
    &= \boxed{0.0010}
\end{align*}
Lastly, $p = \frac{1}{\tau} = \tau^{-1}$. Since this is the special multiplicative case as derived in Eq. (\ref{eq:error_propagation_pure_multiplicative}),
\begin{align*}
    \delta_p &= p \frac{\delta_{\tau}}{\tau} \\
    &= 0.9096 \times \frac{0.0103}{1.0994} \\
    &= \boxed{0.013}
\end{align*}