\section{Appendix: Effective Stiffness from Load-Deflection Data}
\label{app:spring_stiffness}
This appendix outlines the calculations used to determine the experimental effective stiffness of the system. The effective stiffness is determined by measuring the deflection of the spring under a known load. The stiffness was determined by a linear regression through the origin of the load-deflection data. The data is shown in Table \ref{tab:load_deflection}.

\subsection{Effective Stiffness Calculation}
\begin{table}[H]
    \centering
    \caption{Load-Deflection Data}
    \label{tab:load_deflection}
    \begin{tabular}{ccccc}
    \toprule
        Mass & Force & Initial Position, $a$ & Final Position, $b$ & Deflection, $x$ \\
        (g) & (N) & (cm) & (cm) & (m) \\
        \midrule
        0 & 0.000 & 91.1 & 91.1 & 0.00 \\
        50 & 0.491 & 91.1 & 87.4 & 0.0370 \\
        100 & 0.981 & 91.1 & 83.8 & 0.0730 \\
        150 & 1.47 & 91.1 & 80.3 & 0.108 \\
        200 & 1.96 & 91.1 & 76.4 & 0.147 \\
        250 & 2.45 & 91.1 & 73.3 & 0.178 \\
        300 & 2.94 & 91.1 & 69.0 & 0.221 \\
        \bottomrule
    \end{tabular}
\end{table}
Sample calculations for the effective stiffness are shown for the 50 g mass. The deflection was found by

The spring was subject to gravitational force from the applied mass. The force was found by 
\begin{align*}
    F &= mg \\
    &= \qty{50e-3}{\kilo\gram} \times \qty{9.81}{\meter\per\second\squared} \\
    &= \qty{0.491}{N}
\end{align*}
The deflection was found by
\begin{align*}
    x &= a - b \\
    &= \qty{91.1}{\centi\meter} - \qty{87.4}{\centi\meter} \\
    &= \qty{0.0370}{\meter}
\end{align*}
Next, a linear regression was applied to the data to determine the effective stiffness. From Excel, 
\begin{table}[H]
    \centering
    \caption{Linear Regression Data}
    \label{tab:linear_regression}
    \begin{tabular}{cc}
    \toprule
        Parameter & Value \\
        \midrule
        Spring constant, $k_e$ & $\qty{13.5}{\newton\per\meter}$ \\
        Standard error, $S_k$ & $\qty{0.082}{\newton\per\meter}$ \\
        $R^2$ & 0.9998 \\
        \bottomrule
    \end{tabular}
\end{table}
So, the effective stiffness of the spring is
\begin{empheq}[box=\fbox]{align*}
    k_e &= \qty{13.5}{\newton\per\meter}
\end{empheq}

\subsection{Effective Stiffness Error}
\subsubsection{Error Propagation Derivation}
\label{sec:error_propagation_derivation}
To be thorough, the error propagation formula will be derived for completeness. 

If we know how a quantity of interest depends on other, directly measurable quantizes, it is posable to estimate the uncertainty of this "output" quantity based on the uncertainties in the measured quantizes. For example, we can calculate the uncertainty associated to a volume based on the uncertainty of the measurement of the individual dimensions.

Consider as results, $R$, which is a function of $n$ variables, $x_1, \ldots, x_n$:
\begin{align*}
    R &= f(x_1, \ldots, x_n)
\end{align*}
If the individual measurands, $x_i$, have an associated uncertainty $w_{x_i}$, what is teh uncertainty of $w_R$ of the result R?

Defining $x:= (x_1, \ldots, x_n)$, and $x_m := (x_{m, 1}, \ldots, x_{m, n})$, perform the Taylor series expansion of $R= f(x)$ about the point $x = x_m$, taking $x_i - x_{m, i} = w_{x_i}$:
\begin{gather*}
    R = f(x_m) + \frac{\partial f}{\partial x_1} \bigg|_{x = x_m} \underbrace{(x_1 - x_{m, 1})}_{w_{x_1}} + \ldots + \frac{\partial f}{\partial x_n} \bigg|_{x = x_m} \underbrace{(x_n - x_{m, n})}_{w_{x_n}} + \text{H.O.T.} \\
    \underbrace{R - f(x_m)}_{w_R} = \frac{\partial f}{\partial x_1} \bigg|_{x = x_m} w_{x_1} + \ldots + \frac{\partial f}{\partial x_n} \bigg|_{x = x_m} w_{x_n} + \text{H.O.T.}
\end{gather*}
The higher-order terms contain quadratic terms $w_{x_i}w_{x_j}$, cubic terms $w_{x_i}w_{x_j}w_{x_k}$, and so on. Assuming the individual uncertainties $w_{x_i}$ are small,  we can take these higher-order terms as zero, giving 
\begin{align*}
    w_R &= \sum_{i=1}^n \bigg|\frac{\partial f}{\partial x_i} \bigg|_{x = x_m} w_{x_i} \bigg|
\end{align*}
However, this is the worst-case uncertainty, and is an overly conservative estimate. A better estimate is to use the root of sum of squares
\begin{align}
    w_R &= \sqrt{\sum_{i=1}^n \left[\frac{\partial f}{\partial x_i} \bigg|_{x = x_m} w_{x_i}\right]^2} \label{eq:error_propagation}
\end{align}
If the confidence levels associated to the individual uncertainties $w_{x_i}$ are all identical (for instance 95\%), the confidence level of the result $w_R$ will be the same. 

The key assumption behind RSS is that the set of measured variables $x_1, \ldots, x_n$ are \textbf{statistically indecent}. If this is not the case, a different formulation needs to be used. Also note that all uncertainties $w_{x_i}$ need to be small such that the first-order Taylor series approximation holds. 

Consider the case where the result $R$ is dependent only on the product of the measured variables, $x_1, \ldots, x_n$ with associated uncertainties $w_{x_1}, \ldots, w_{x_n}$ as 
\begin{align*}
    R = C x_{1}^{c_1} x_{2}^{c_2} \ldots x_{n}^{c_n}
\end{align*}
where $C$ and $c_1, \ldots, c_n$ are constants. In this case, the RSS formula gives
\begin{align}
    w_{R} = \sqrt{\left(C c_1 x_{1}^{c_1 - 1} w_{x_1}\right)^2 + \ldots + \left(C c_n x_{1}^{c_1} x_{2}^{c_2} \ldots c_n x_{n}^{c_n - 1} w_{x_n}\right)^2} \nonumber \\
    \implies \frac{w_{R}}{|R|} = \sqrt{\left(\frac{c_1 w_{x_1}}{x_1}\right)^2 + \ldots + \left(\frac{c_n w_{x_n}}{x_n}\right)^2} \label{eq:error_propagation_pure_multiplicative}
\end{align}

Recall from MEC E 300 that the error for the coefficients for the form $f(x) = ax +b$ can be found by finding standard error of $a$ and $b$ by first finding 
\begin{align*}
    S_{y,x} &= \sqrt{\frac{\sum (y_i - a x_i - b)^2}{n-2}} \\
\end{align*}
then,
\begin{align*}
    S_a &= S_{y,x} \sqrt{\frac{1}{\sum (x_i - \bar{x})^2}} \\
    S_b &= S_{y,x} \sqrt{\frac{\sum x_i^2}{n \sum (x_i - \bar{x})^2}}
\end{align*}
then with a given confidence of $1 - \alpha$, the uncertainties are 
\begin{align*}
    a \pm t_{\alpha/2, n-2} S_a \\
    b \pm t_{\alpha/2, n-2} S_b
\end{align*}
The uncertainty of the slope is then \cite{wheeler_ganji}
\begin{align}
    w_a = t_{\alpha/2, n-2} S_a \label{eq:slope_error}
\end{align}

\subsubsection{Error Propagation for Effective Stiffness}
Given a confidence level of 95\%, the t-distribution value was determined by
\begin{align*}
    \alpha/2 &= \frac{1 - 0.95}{2} = 0.025 \\
    n - 2 &= 7 \\
    t_{\alpha/2, n-2} &= 2.5706
\end{align*}
By Eq. (\ref{eq:slope_error}), the error in the effective stiffness with a confidence level of 95\% is
\begin{align*}
    \delta_{k_e} &= t_{\alpha/2, n-2} S_k \\
    &= 2.5706 \times \qty{0.082}{\newton\per\meter} \\
    &= \boxed{\qty{0.21}{\newton\per\meter}}
\end{align*}



