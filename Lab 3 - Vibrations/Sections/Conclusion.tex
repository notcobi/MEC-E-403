\section{Conclusion}
This experiment served to analyze a single degree of freedom spring-cart system undergoing different forms of vibration. This analysis used data from free, forced, and damped vibration tests, and compared the results from these tests against their theoretical counterparts.

The effective stiffness, $k_e$, of the system was determined to be $13.5 \pm 0.2$ N/m, derived from the load-deflection data. This result was obtained through the application of known masses to the spring, with a coefficient of determination $R^2 = 0.9998$ for a linear line forced through the origin. The plot is consistent with Hooke's Law, which states that the force exerted by a spring is linearly proportional to the displacement of the spring. The high coefficient of determination is evidence of the high linearity of the data.

The natural frequencies, $p$, were determined using $k_e$ and the effective mass, $m_e$, of the system. The natural frequencies obtained were as follows: $p_{\text{small, sys}} = 1.1317$ Hz, $p_{\text{small, no sys}} = 0.9096$ Hz, $p_{\text{big, sys}} = 0.8576$ Hz, and $p_{\text{big, no sys}} = 0.7513$ Hz. These are the natural frequencies for the small and big carts, with and without the measurement system attached. 

Next, the free vibration data was also used to determine the natural frequencies of the system by displacing the cart and measuring the time it takes to complete 10 full oscillations. The natural frequencies obtained were as follows: $p_{\text{small, sys}} = 1.1317 \pm 0.013$ Hz, $p_{\text{small, no sys}} = 0.9096 \pm 0.009$ Hz, $p_{\text{big, sys}} = 0.8576 \pm 0.012$ Hz, and $p_{\text{big, no sys}} = 0.7513 \pm 0.016$ Hz. These natural frequencies were derived from experimental measurements, comparing observed free vibration data with theoretical expectations. Overall, the experimental results closely matched the results obtained by using the effective stiffness and effective mass, being within 2\% of the free vibration results.

The average damping ratio, $\zeta$, of the system was calculated from the decaying oscillations observed in the free vibration data. Two trials were performed for both the small and big carts, without the measurement system attached. The damping ratio for the small cart was found to be $\zeta_{\text{small}} = 0.025$ and for the big cart was found to be $\zeta_{\text{big}} = 0.018$. Software was used to track the amplitude of the cart. These values were obtained by selecting several pairs of consecutive maxima along the oscillations away from the end of the oscillations. The damping ratio did not stay constant, which is not consistent with the linear viscous damping model. In fact, the damping ratio increased exponential-like as it approached rest.  This could be due to non-linear damping effects, such as torsional friction in the pin, or friction in the air track. Generally, the linear viscous damping model provided a reasonable approximation of the system's behavior when the amplitude was large, but became less accurate as the amplitude decreased.

A forced vibration test was performed to determine the dynamic magnification factor (DMF) of the system. The DMF was calculated using the measured displacements for both masses, plotted against the frequency ratio $\omega/p$. The DMF closely followed theoretical expectations, with slight deviations observed, particularly in the out-of-phase region. These discrepancies could be attributed to neglected inertial effects and friction in the system. However, overall, the agreement between theory and experiment was satisfactory.

The effects of the pulley measurement system on the natural frequency calculation were examined by considering the inertia of the pulleys. By drawing a free-body diagram and analyzing the system, it was determined that the effective mass of the system increased when the measurement system was attached. The relationship was determined to be $m_e = m_{\text{cart}} + 0.5 m_{\text{pulley 1}} + 0.5 m_{\text{pulley 2}}$. The addition of the pulleys adds an effective mass of 0.1368 kg to the system. The correction required in the natural frequency calculation highlighted the importance of accounting for inertial effects in experimental setups.

Finally, the effective mass of the system was determined from free vibration data. The effective mass was determined to be $0.4099$ kg for the small cart and $0.5818$ kg for the big cart. This was determined by comparing the natural frequency of the system with and without the measurement system attached. The effective mass determined by free vibration closely matched theoretical values derived from mechanics within a relative error of $2.09\%$ for the small cart and $0.25\%$ for the big cart. The low relative error indicates that the effective mass was determined accurately. The effective mass from the vibration data was found to be higher than the theoretical value, which could be due to neglecting the mass of the springs and cables, and other inertial effects such as the friction from the air track and the torsional friction in the pins of the pulleys. Overall, the experimental results provided valuable insights into the behavior of the system, emphasizing the importance of comprehensive analysis and consideration of all relevant factors in experimental design.

\subsection{Technical Recommendations}
The experiment was successful in determining the effective mass of the system. However, there are several areas where improvements could be made. The damping ratio was found to increase as the amplitude decreased, which is not consistent with the linear viscous damping model. Further investigation into the causes of damping fluctuations could provide valuable insights into system dynamics. 

Additionally, the DMF was found to deviate slightly from theoretical expectations, particularly in the out-of-phase region. This discrepancy could be attributed to neglected inertial effects and friction in the system. Future experiments could focus on reducing these discrepancies by accounting for additional inertial effects and friction in the system. 

The uncertainty of the natural frequency could be reduced by using an the data acquisition system to measure the time it takes to complete 10 cycles. This would reduce the uncertainty in the period calculation, which would reduce the uncertainty in the natural frequency calculation.

Finally, the effective mass was found to be higher than the theoretical value, which could be due to neglecting the mass of the springs and cables, and other inertial effects. Future experiments could aim to reduce this discrepancy by accounting for these additional inertial effects in the experimental setup.