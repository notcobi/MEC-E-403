
\section{Introduction}
Vibrations are ubiquitous in mechanical systems, manifesting in various scenarios such as pendulums, strings on pulleys, and ground-level buildings [1]. Identifying and understanding these vibrations is crucial for ensuring the safety and structural integrity of objects and materials in use. Vibrations can generally be categorized into forced and free types. Forced vibrations persist due to external forces acting on the system throughout the oscillation period, while free vibrations gradually lose energy over time [2]. In this laboratory experiment, both forced and free vibration setups are explored using single degree of freedom (SDOF) systems, serving as close approximations to real-life scenarios.

This lab report aims to explore fundamental vibration theories through experimentation on free and forced responses of small and large masses. The primary objectives include obtaining experimental values for parameters such as damping ratio, natural frequency, and dynamic magnification factor (DMF), while also investigating the effects of pulleys and comparing experimental results to theoretical predictions. Understanding the impacts of free and forced vibrations on mechanical systems is crucial for various engineering applications.

The experimentation begins with the calculation of the effective spring constant by attaching various weights to a cart and measuring displacement for each additional mass added. Subsequently, damping ratio and natural frequency are determined through free vibration trials. The theoretical natural frequency is derived from the spring constant and cart masses, while the experimental natural frequency is calculated by counting the number of cycles the cart undergoes within a specific time frame during free vibration. The calculated damping ratio is compared to the linear viscous damping model to assess its ability to represent energy dissipation in the air track system.

Next, DMF values are calculated during forced vibration tests. Experimental DMF data is collected using a data acquisition system (DAS), where the system undergoes vibrations at four frequencies both higher and lower than the resonance. The experimental DMF data is then compared to theoretical DMF values to analyze their correspondence during in-phase and 180 out-of-phase periods.

Further analysis involves calculating the forces acting on the mass cart and pulley using free-body diagrams. These force equations allow for corrections to the natural frequency values by adjusting the main mass values. Finally, the masses of the large and small pulleys are calculated and compared to nominal values.

The structure of this lab report includes detailed descriptions of the experimental procedure and equipment used, followed by a theoretical background section and analytical results and discussions on the findings. The report concludes with a summary of findings and technical recommendations for improving the experiment in the future.